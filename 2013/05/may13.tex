% Created 2013-05-31 Fri 15:35
\documentclass[11pt]{article}
\usepackage[utf8]{inputenc}
\usepackage[T1]{fontenc}
\usepackage{fixltx2e}
\usepackage{graphicx}
\usepackage{longtable}
\usepackage{float}
\usepackage{wrapfig}
\usepackage{soul}
\usepackage{textcomp}
\usepackage{marvosym}
\usepackage{wasysym}
\usepackage{latexsym}
\usepackage{amssymb}
\usepackage{hyperref}
\tolerance=1000
\providecommand{\alert}[1]{\textbf{#1}}

\title{may13}
\author{Nicolas}
\date{\today}
\hypersetup{
  pdfkeywords={},
  pdfsubject={},
  pdfcreator={Emacs Org-mode version 7.8.11}}

\begin{document}

\maketitle


\section*{Wednesday 8th}
\label{sec-1}
\subsection*{Automation programming}
\label{sec-1-1}
\subsubsection*{10:20 Server-client model}
\label{sec-1-1-1}

I've built a generic p7888 client as described in the last entry of
April's logs. There have been plenty of bugs and fixes since then, and
it's still not in great working order yet, though I haven't tested the
latest iteration. There's a real problem with latency over the
datasocket and resultant synchronisation issues. For simple operations
such as changing the settings, starting and halting the acquisition,
I've managed to sort these issues out using vis that synchronise the
client and server at the appropriate points, but for `live-feed' of
acquisition status or data, I've not yet cracked a working
version. Still, it's not required yet, so I can leave it as a work in
progress. Ultimately I can still ditch the client-server architecture
and have the whole thing running on the server if I want. 
\subsection*{Beamline testing}
\label{sec-1-2}
\subsubsection*{10:35 Re-testing pulsing prechamber}
\label{sec-1-2-1}

Setting up the same beamline tests I went through in early
January. Using a variable-pressure N2 chamber behind the pulse valve
in order to control the amount of flow into the beamline and keep the
pressure resulting from pulsing low. 
\subsubsection*{11:13 Previous tests}
\label{sec-1-2-2}

From the notes of \hyperref[id-4c5ea3d0-f615-4635-972b-0d0adbee38df]{Characterise preliminary N2 chamber} (3rd Jan `13),
the procedure for testing was (from the relevant stage):

\begin{enumerate}
\item Introduce N2 via the leak-valve with the pulse valve closed, making
   sure there is a constant flow of N2 through the valve and down the
   backing line. Pressure in this pre-chamber can then be adjusted via
   the leak valve or the close-off valve for the backing line.
\item Open pulse-valve with a dc voltage to keep it open, and measure the
   pressure inside the beam-line at the second chamber. Adjust
   pressure inside the second chamber using leak valve. The measured
   pressure change can be used to estimate the flux of N2 in pulsed
   mode. Hopefully by that stage we will know a bit more about how
   much N2 flux is required at the trap for succesful photionisation.
\end{enumerate}

Measurements were taken with an N2 regulator pressure of 1 bar, and a
DC valve current of 160 mA (close enough to fully opened, which is at
230 ma, just prefer to use less current). Not sure what voltage that
is, will check earlier notes in a second. -> Datasheet says 20 V. 
\subsubsection*{11:41 Precision valve for the backing line}
\label{sec-1-2-3}

My notes indicate that the precision leak valve used for closing off
the backing line may have been to small i.e. the prechamber pressure
too high. I need to double check this result. The alternative is to
use the main valve behind the belows, but I know that that is very
large and opens quickly. Still, this would give us a lower pressure in
the prechamber when opened.
\subsubsection*{16:15 Statically open pulsing valve}
\label{sec-1-2-4}

Preliminary tests with statically opened valve are complete. I need to
check how similar they are to previous tests, because I don't remember
this behaviour: The valve opens extremely quickly at around 160 mA
(roughly 5V), up to a pressure of around 1E-5 mbar. This basically
indicates a high pressure behind the valve, which is expected. In the
very small range where we are still around 1E-8 mbar (up from 1E-9
mbar with the valve closed), there is no readable change in the ion
pump current, which is the only measure of the pressure in the second
chamber. This is a good confirmation of isolation between the two
chambers. 

Next tests will have to be with pulsing.
\section*{Friday 10th}
\label{sec-2}
\subsection*{Nvidia GPU programming training today}
\label{sec-2-1}
\subsubsection*{09:58 Follow-up}
\label{sec-2-1-1}

Follow-up e-mail from Thomas Nowotny (organiser):

``Hi all,
thanks for coming along today and for the lively session. As promised,
the pdfs of Jeremy's and Timothy's presentations are available here:
\href{http://www.sussex.ac.uk/Users/tn41/nVIDIA-presentations.zip}{http://www.sussex.ac.uk/Users/tn41/nVIDIA-presentations.zip}

The material is also available on the hpc Wiki (thanks, Emyr):
\href{http://www.hpc.sussex.ac.uk}{http://www.hpc.sussex.ac.uk}

The Exercises are still available at
\href{http://www.sussex.ac.uk/Users/tn41/nVIDIA-Exercises.pdf}{http://www.sussex.ac.uk/Users/tn41/nVIDIA-Exercises.pdf}
\href{http://www.sussex.ac.uk/Users/tn41/nVIDIA-Exercises.zip}{http://www.sussex.ac.uk/Users/tn41/nVIDIA-Exercises.zip}

Jeremy asked me whether he could have the list of the email addresses
of attendants. Please let me know if you would \textbf{not} like me to pass
yours on. He promised to not spam you but only use the addresses
sparsely to inform you of similar events to today's. If I don't hear
your objection I will send them to him Monday.

It is encouraging to see so much interest in GPU computing in Sussex
and nearby \ldots{} all the best for now, Thomas.''
\section*{Thursday 11th}
\label{sec-3}
\subsection*{Beamline testing}
\label{sec-3-1}
\subsubsection*{11:20 pulsing experiment}
\label{sec-3-1-1}

Pulsed testing the beamline. Same setup as previously (\hyperref[id-8e2fb259-908c-4369-9586-0ba023a357a6]{Wed. 8th}):
Backing line separated by precision leak-valve which is fully
opened. N2 line separated by precision leak-valve which is opened
minimally, such that the pressure reading on the backing line, which
has a baseline of 1.0E-1 mbar with the N2 closed, now reads 1.5E-1
mbar. 

The pulse width is 100 us and the rep. rate is set internally at 10
Hz (I think). We read the pulse-generator's monitor voltage (V$_m$) as we
increase the pulse height, and monitor the pressure in the first
chamber using the ion gauge (P$_1$), and the second chamber using the ion
pump current (I$_2$). 


\begin{center}
\begin{tabular}{rrr}
 V$_m$ /V  &  P$_1$ /mbar  &  I$_2$ /uA  \\
\hline
       30  &      1.36E-9  &          1  \\
      165  &      1.36E-9  &          1  \\
      170  &      1.50E-9  &          1  \\
      175  &      1.95E-9  &          1  \\
      180  &      2.80e-9  &          1  \\
      185  &       4.9E-9  &          1  \\
      190  &      9.70E-9  &          1  \\
      195  &      1.72E-8  &          1  \\
      200  &      3.25E-8  &          1  \\
      205  &      6.90E-8  &          2  \\
      210  &      1.75E-7  &          4  \\
\end{tabular}
\end{center}



This set of measurements seems reasonable. It shows that we can reach
a regime where there is some measurable load on the ion pump, which
should be a significant no of molecules. We have a long range below
that as well, which might be what we want, if we have a high enough PI
efficiency. The results don't quite match those of the same kind of
test previously (\hyperref[id-2e3d7b82-77d8-4e7f-82b9-6300c3e6dcfb]{January}), but this is likely due to my having swapped
around the precision valves which control the backing and N2 lines,
which are a different model. It seems to me that the valve that now
controls the backing line has a smaller maximum opening, resulting in
a higher pressure in the prechamber and a higher pressure in
chamber 1. 
\subsubsection*{11:41 pressure buildup?}
\label{sec-3-1-2}

To avoid the effects of long term buildup of pressure in the second
chamber, due to low pumping speed of the ion pump, I made sure to ramp
down the valve voltage to close it after each reading - in order to
give the ion pump enough time to completely pump out. Doing this,
there didn't to actually be any long-term build up for any of the
readings since the pressure returned almost instantaneously to the
baseline. However this does not tell us that there is no buildup at
all, since on the scale of the rep. rate (10 Hz) there could well be
some uncleared molecules. 
\subsubsection*{16:28 prechamber configuration}
\label{sec-3-1-3}

I've looked again at the notes of January to understand better the
work that I did then. The conclusions I came to appear to be that the
pressure in the second chamber (beamline) was too high without a
prechamber, and too high with the precision leak-valve separating the
prechamber from the backing line, since there is too small an aperture
from the backing line to pump through.

When not using a precision leak valve, but a larger standard valve,
the relative pressure in the prechamber is much smaller, and so is the
pressure in the second chamber during pulsing. So small in fact that
it was not measured by the ion gauge. We think that this might be
okay, since we in theory only need a few molecules, but it would have
been nice to hit the range in between this and the sans-prechamber
configuration. Unfortunately we have no intermediate size valves to do
this with.

Instead I think the best configuration to start with would be the one
with the large standard valve. According to my measurements the
optimal configuration is with that valve fully open, and the N2
leak-valve open so far as to load the backing pump up to around 0.5
mbar. This is the maximum prechamber pressure we can get in this
configuration, before the load on the backing gets so high that the
turbo begins to struggle, and the second chamber pressure begins to
rise via that mechanism. 

During tests, when the pulsed valve was held statically open with 160
mA current, the pressure in the second chamber under the above
conditions was 7.5E-8 mbar (35 uA on ion pump). The baseline pressure
(valve closed) was 1.3E-8 mbar (6 uA on ion pump). We should be able
to calculate the flux of molecules from these numbers.

If it turns out that this configuration does not give us enough
molecules at the target, then we can remove the prechamber and try
again. 
\section*{Monday 13th}
\label{sec-4}
\subsection*{Beamline testing}
\label{sec-4-1}
\subsubsection*{11:04 Remade prechamber measurements}
\label{sec-4-1-1}

Previous measurements: \hyperref[id-2e3d7b82-77d8-4e7f-82b9-6300c3e6dcfb]{January}
Today's measurements:
Testing how much N2 pressure we can get into the prechamber before the
load on the backing pump starts to affect the pressure in the second
chamber (measured by the ion pump current).


\begin{center}
\begin{tabular}{rrr}
 P$_b$ /mbar  &  P$_1$ /mbar  &  I$_2$ /uA  \\
\hline
      6.5E-2  &       1.2E-9  &          0  \\
      2.0E-1  &       1.3E-9  &          0  \\
      2.5E-1  &       1.6E-9  &          0  \\
      3.0E-1  &       2.6E-9  &          0  \\
      3.5E-1  &       5.0E-9  &    0.5 (N)  \\
\end{tabular}
\end{center}



N: Flashing between 0 and 1.
These results are roughly in line with what we saw before. P$_1$
pressure starts to rise early at around 3.0E-1 mbar backing pressure,
and Ion pump current shortly after. Ion pump current is rising sooner
than in previous experiments, but this is likely just because the
pressure in that chamber is smaller than before and therefore we are
more sensitive to changes.

The next measurements taken also agree with those found at the entry
linked above. With the current sensitivity of pressure measurements in
the second chamber, no change in pressure is detectable during pulsed
mode, even at full opening of the valve. There is an increase in
pressure in the first chamber of course, and so we do have a good
chance of there being a proper N2 beamline. It's just going to take
testing to tell whether it is dense enough in this configuration. The
numbers are in the entry of my notebook for today. 
\subsubsection*{12:15 Next steps}
\label{sec-4-1-2}

The next step is to open up the gate-valve separating the beamline
from the ion trap. We must wait for Amy to get micromotion
compensation working again so that she can move on to the optics and
trapping - We should only open the beamline to the trap shortly after
confirming normal trapping. 

Once they are open to each other and we have trapped, we should redo
the measurements I've made above, looking again at what backing
pressure starts to affect the pressures in the main chambers,
specifically. 

Then, once the optics for PI are aligned, we must work on the timing
of the valve pulsing with respect to laser trigger. There is a delay
circuit built in to the pulse generator, so we shouldn't have a lot of
trouble with this stage. 
\subsection*{Automation programming}
\label{sec-4-2}
\subsubsection*{16:24 In good shape}
\label{sec-4-2-1}

The server-client p7888 automation software is finally in good
shape. Overall the redesign I had been exploring, featuring
client-server synchronisation vis, turned out well after rooting out
the standard first-deployment bugs. 

So far this is only a test vi, but in fact it can be used to make any
of the standard measurements we normally make, and can be
extended/rewritten to do probably everything that we need for the real
experiment. 

Now that the guts should be more or less settled, I should write up
documentation and labview help-files, so that people can build their
own software with it.

I should also now speak with the others about laying down plans for
the real experiment automation. Also, maybe some speed tests would be
interesting. 
\section*{Wednesday 15th}
\label{sec-5}
\subsection*{N2 Photionisation scan}
\label{sec-5-1}
\subsubsection*{13:38 Starting up}
\label{sec-5-1-1}

This is my new task for the mean-time. Matthias has replaced what
might have been a faulty channeltron for detection, and optimised most
of the dye laser setup. The autotracker still needs some work
apparently, which I'll ask Jack about. 

The current pulse energy in UV is 1 mJ and there's some losses after
going through the N2 chamber, leaving us with 500 uJ. 
\subsection*{State detection experiment planning}
\label{sec-5-2}
\subsubsection*{13:41 Points of consideration}
\label{sec-5-2-1}

We had a quick meeting this morning to discuss the points of the
experiment that still need planning and preparation.

\begin{enumerate}
\item Photoionisation appears to be working, but we would like
   confirmation of the frequency of the ground state photoionisation
   transition, which was just outside of the range of the only scan we
   took.
\item N2+ spectroscopy (Will's experiment) looked fine, but might be able
   to be improved, since the frequency of the transition seen was
   around 100 GHz out from literature values. We could just increase
   our planned detuning to mitigate any issues with this, but Matthias
   has some plans for additions to the experiment that might fix this
   problem.
\item The issue of the N2-Ca crystal configuration needs to be considered
   carefully. If the `phonon-laser' trick we are considering works,
   then it is not an issue, but I don't know if it will work. If it
   doesn't, then we must make sure that the crystal is in the same
   configuration every time the dipole force is applied so that a
   change in dipole force phase does not destroy correlations in the
   analysis. This requires that we track the position of the ions in
   the two-ion string, and either force it to remain the same, or wait
   til it reconfigures before applying another dipole force. 

   There's no way, with a two-ion string, to tell the difference
   between configurations with just the PMT and it seems unlikely that
   there will be room for a camera in the setup. One alternative is to
   use a three-ion string, with two calcium ions and one N2. This way
   the configurations have different secular frequencies, and thus a
   change in configuration could be detected.

   The other alternative would be, using a two-ion string, to make
   sure that in either position the N2 ion would be at the antinode of
   the dipole force. It's worth thinking about how we could do that
   practically, I think.
\item The pulse laser and the pulse valve need to be synchronised. he
   problem here being that the pulse laser does not output a
   synchronising signal until it fires, and it does not fire at a
   fixed interval after it has been told to (only at the next point
   which matches with its internal 10 Hz trigger). This makes the
   firing time difficult to predict, and since the pulse valve needs
   to be opened at some time \emph{before} the laser fires, so that
   molecules have time to reach the trap, we have a synchronisation
   issue.

   Approach number one is to check if we can somehow access a signal
   from the laser's internal timer - it's not immediately accesible
   according to Matthias, but he hasn't checked thoroughly. 

   Approach number two is to have two pulsed from the laser per pulse
   from the valve - one before, to give us the timing, and one after
   which ionises the released molecules.
\end{enumerate}
\section*{Friday 17th}
\label{sec-6}
\subsection*{N2 Photionisation scan}
\label{sec-6-1}
\subsubsection*{13:26 Still haven't started}
\label{sec-6-1-1}

The setup is roughly optimised, according to Matthias. The
autotracking needs to be set for whatever wavelength scan range we are
working on, so this is something I need to speak with Jack about when
it comes to actually making a scan.

Also, Matthias is fairly worried about there only being 1 mJ pulse
energy despite his best efforts to optimise. The current thinking is
that the beam shape is too bad. He has contacted Radiant Dyes (the
company who made the laser), for advice and the possibility of a
technician.

On top of power issues, a bad beamshape makes the auto-tracker not
work very weel. Matthias' suggestion is that we just not use it, and
make scans small enough that there is not a big loss of power. The
power fluctuations we do see however, should be monitored and used to
obtain a normalised ionisation spectrum.
\section*{Monday 20th}
\label{sec-7}
\subsection*{N2 Photoionisation}
\label{sec-7-1}
\subsubsection*{09:43 Alignment}
\label{sec-7-1-1}

I got the dye laser up and running on Friday afternoon. I've installed
a flipper mirror so that it can be used in Amy's setup when the time
comes. This mean that I had to realign the laser through the chamber,
and it now goes through losing half power (the same as when it was
aligned through the chamber before, according to Matthias). When I
finished up on Friday afternoon, there was 1.1 mJ coming out of the
laser, and roughly 550 uJ after the PI chamber.

Next up is a quick scan with no N2.
\section*{Tuesday 21st}
\label{sec-8}
\subsection*{N2 Photoionisation}
\label{sec-8-1}
\subsubsection*{11:36 Channeltron woes}
\label{sec-8-1-1}

I've spent the last day setting up the spectroscopy experiment, but
just this morning found that the channeltron performance appears to be
degrading very quickly.

Matthias originally told me that the c.tron runs at 2 kV. When I ran
it at this voltage I saw no counts. Turning up the voltage to around
2.2 kV I saw some counts for a short time, before they were gone
again. It seems that the threshold voltage for getting counts on the
device is creeping up, with counts last seen at 2.5 kV. I don't want
to keep tracking that in case I properly blow the system (I think the
limit is 3 kV).

It might be a good idea, just as a test, to switch the polarity on the
device and look for photoelectrons from the laser, since there will be
a higher background signal in this situation. 

The other, perhaps more sensible alternative is to introduce some N2
and have some ions, just to see if I'm just not seeing counts because
the background is too low. Although actually\ldots{} the count-rate from N2
ionisation is supposed to be quite low compared to the background
ionisation events when the laser is turned on, and I've already tested
with the laser turned on so I guess there's not really any point in
that. 
\subsubsection*{13:49 broken connection?}
\label{sec-8-1-2}

Matthias has suggested that a connection to the c.tron could be
broken. If there was only a small breakage, then the c.tron could
still be coupled capacitatively, which would explain why it works for
just a short period of time. I've tested the voltage at the
feedthrough and it seems fine. It's possible still that there is a
broken connection inside the chamber however.
\section*{Wednesday 22nd}
\label{sec-9}
\subsection*{N2 Photoionisation}
\label{sec-9-1}
\subsubsection*{09:43 Electronics problem}
\label{sec-9-1-1}

Yesterday I opened the chamber to check for broken connections and
found none. The chamber is closed again and appeared to be pumping
down to a reasonable pressure again despite reusing the main gaskett
and not baking (high E-7s after only a half-hour or so). 

Testing the electronics again after pumping down however, I noticed
that the channeltron \emph{does} work as long as the power supply for the
discriminator is turned off. Unfortunately this wasn't checked before,
as I felt it would be enough just to unplug the output of the c.tron
from the input of the discriminator. It seems as if, then, there is a
problem perhaps with grounding of that supply. Perhaps testing power
and ground connections is the best approach from here. 
\section*{Thursday 23rd}
\label{sec-10}
\subsection*{N2 Photionisation}
\label{sec-10-1}
\subsubsection*{09:20 No electronics problem}
\label{sec-10-1-1}

I don't think there's a problem with the electronics any more. I had
unplugged the 50 ohm termination at some point and that was causing
the noise spike from the laser trigger to extend over the time period
where we expect to see the arrival of ions on the c.tron. The
discriminator itself is 50 ohm, so the issue with that was not an
issue, just it running properly - the reason I didn't recognise this
is that I had no idea that the real c.tron triggers were so short, and
difficult to see (\~{}ns).
\subsubsection*{09:22 Scanning issues}
\label{sec-10-1-2}

I'm now trying to make spectra in the regions beyond where Jack's
spectra finished. The auto-tracker that compensates the angle of
incidence on the doubling crystal as the wavelength changes is
apparently a little hard to deal with, if it's even working at all
(see Jack's scan for power variations). Thus I'm looking to make a
scan without using it at all. The idea is to set the tracker position
so that the output power of the dye-laser is maximum at the centre of
the scan, and then start a scan that goes from below that centre to
above it. 

Unfortunately the vi that sets the wavelength for the scan appears to
mess up the position of the tracker at the beginning of the scan. The
position of the tracker at the beginning of the scan depends on where
it is initially pre-optimised to in a way that I can't work out. I'm
currently in the process of working out how to use the vis that come
from the company to change the wavelength, instead of the one in the
scan vi built by Jack (who can't remember where he got the vi that
changes the wavelength currently). 
\section*{Friday 24th}
\label{sec-11}
\subsection*{N2 Photoionisation}
\label{sec-11-1}
\subsubsection*{09:54 Scanning fixed}
\label{sec-11-1-1}

Yesterday's scanning issues were eventually cured. In fact, the change
in power was not due to a moving motor, and wasn't a changing power at
all. Instead the vi that was averaging the power readings being
received was buggy (beware of the labVIEW ``elapsed time'' express
vi!). I fixed it, and now the power changes as we would expect it to.

Ready to take some scans, in principle.
\section*{Tuesday 28th}
\label{sec-12}
\subsection*{N2 Photionisation}
\label{sec-12-1}
\subsubsection*{12:02 More/same channeltron problems}
\label{sec-12-1-1}

I appear to be seeing the same channeltron problems as I was seeing
before (\hyperref[sec-8-1-1]{11:36 Channeltron woes} to \hyperref[sec-10-1-1]{09:20 No electronics problem}). That
time things seemed to magically fix themselves, but I'm not seeing any
signal any more and its not magically fixing itself. 

Again I can see ions behaving in the way we would expect with respect
to the capacitor voltage, but only when I shine the laser deliberately
so that it is hitting the side of the N2 enclosure. As soon as we have
good clearance (reasonable energy throughput), there are zero counts
on the c.tron (not even the expected background counts).

First thing in the morning I tried bringing the chamber up to 1E-3
mbar of N2, to see if that would discharge anything that had charged
up and was disrupting the c.tron operation, but that didn't help at
all. 

N.B. I believe that these issues are very close to what Jack was
seeing before he had to move on to fibre shooting. We're no closer to
solving them at the moment I'm afraid. 

Matthias made mention that at one point he had seen difficulty using
that particular discriminator with the c.tron, and found that
unplugging it allowed him to observer pulses on the c.tron. I have
just tried that (replacing the disc. with a 50 ohm termination) and
found no trace of a signal in the region of interest. 
\subsubsection*{12:12 Scan with scatter?}
\label{sec-12-1-2}

I suppose that one thing we could do is to take a scan with the laser
pointing towards the region where we do see counts on the channeltron
from scattering off of the enclosure. Perhaps if we can see a
spectroscopic signal from N2 over the noise, we can use that to get
some signal optimisation going, and help us find our way towards a
clean spectrum.
\section*{Wednesday 29th}
\label{sec-13}
\subsection*{N2 photionisation}
\label{sec-13-1}
\subsubsection*{09:51 No spectroscopic properties in the scatter \textbf{:Log:}}
\label{sec-13-1-1}

Half-expected: there are no spectroscopic properties in the
scatter. Besides that, the signal immediately after the laser pulse
looks like its overloading the c.tron, so I should probably not
continue to pursue this course. 

At the end of the day I tried letting the pressure of N2 in the
chamber get up to atmosphere, in case that discharged or cleaned any
of the surfaces that could have been messing things up. A quick test
showed no signal on the c.tron, but further testing is required.

If that doesn't work, I wonder if its possible that the channeltron's
electrode surfaces are contaminated, perhaps with pump oil? It's
possible that this could reduce the signal that we see.
\subsubsection*{11:49 Spectrum-like features observered, need confirmation \textbf{:Log:}}
\label{sec-13-1-2}

Strong spectrum-like features can be observed if the c.tron voltage is
turned up to its maximum (3 kV) and the discriminator is turned down
low - This is because even at 3kV the c.tron pulses are
indistinguishable, to the eye, from noise. This was something that had
been misleading me up until now, as I thought that the c.tron just
wasn't working.

Essentially the discriminator is triggering off of noise a lot of the
time, but there are also large count-rate peaks that span three or
four points of a spectrum to give us what look like strong ionisation
peaks. The clearest peak seen had a count rate of 120 Hz on a background
of around 10 Hz.
\section*{Thursday 30th}
\label{sec-14}
\subsection*{N2 Photoionisation}
\label{sec-14-1}
\subsubsection*{10:16 Partial spectra obtained, need more data \textbf{:Log:}}
\label{sec-14-1-1}

The spectrum-like features observed in the last entry are confirmed to
be N2 photoionisation peaks. They follow Jack's previous spectra and
have been reproducible.

The autotracker is \emph{not} on, because it has not been reported to be
working well, and would probably take more work than making small
scans and setting the crystal angle manually each scan. I am taking
small scans over 0.08 nm and recording the power of the laser
alongside count-rates to make a compensated scan spectrum.

Before taking more scans though I need to change the dye in the laser
and re-optimise for output power.
\subsubsection*{15:26 Dye mixing procedure \textbf{:Method:}}
\label{sec-14-1-2}

\begin{enumerate}
\item Mixture: 0.4 g coumarin, 500 ml ethanol
\item Stir and break apart clumps of coumarin
\item Ultrasonic bath for 10-20 mins, until mixed
\item Separate into two beakers, 2x250 ml
\item Top one beaker up to 500 ml the other to 800 ml
\item 500 ml goes into oscillator (RHS, smaller bottle)
\item 800 ml goes into amplifier (LHS, larger bottle)
\end{enumerate}
\section*{Friday 31st}
\label{sec-15}
\subsection*{N2 Photionisation}
\label{sec-15-1}
\subsubsection*{11:55 Failed aligning/optimisation \textbf{:Log:}}
\label{sec-15-1-1}

After changing the dye yesterday I found I could not reoptimise the
laser to give a decent power at the output. I've asked Jack for advice
and help, but we didn't manage anything. Now Matthias is having a
look. 

I will try and plot some of the data I managed to get on Wednesday
(hopefully not too much of it was overwritten).
\subsubsection*{15:25 Spectrum analysis \textbf{:Data:}}
\label{sec-15-1-2}

I've analysed the data taken on Wednesday the 29th (\hyperref[sec-13-1-2]{11:49 Spectrum-like features observered, need confirmation}). 

The data itself can be found at this link:
\href{file:///home/nicolas/Documents/logs/2013/05/N2PI/29-05-13}{file:\~/Documents/logs/2013/05/N2PI/29-05-13} 

Where the datasets that are junk have been marked with an underscore
(e.g. ``N2scan10.txt\_''). These are typically scans that I accidentally
overwrote with power calibration data. 

\href{file:///home/nicolas/Documents/logs/2013/05/N2PI_spec.png}{N2 photoionisation spectrum}

The figure above shows all of the individual spectra stitched into one
plot. 
\subsection*{Hanle dip simulation}
\label{sec-15-2}
\subsubsection*{11:56 Revisiting simulation for the experiment/paper \textbf{:Simulation:}}
\label{sec-15-2-1}

Matthias has also asked me to revisit some of the simulations done for
the Hanle dip/magnetic field gradient experiment.

Specifically he'd like to me to produce a plot of how fluorescence
changes for varying magnetic fields, with different polarisations of
cooling and repumper. This would allow us to map what we've seen in
the experiment to a magnetic field, in case we were to write a paper
on the experiment.

He'd also like a sanity-check for the experiment, which is to check
that as the B-field increases, the sensitivity of the fluorescence to
B-field modulation should decrease.

The last thing is that he wanted to me to take another look at the
B-field modulation frequency response drop-off. In the experiment we
were seeing a fast dropoff of the fluorescence modulation response,
whereas in the simulation I found that it was only dropping off at
around the frequency of the spontaneous emission. One suspicion is
that this could have been for Rabi frequencies that matched the
spontaneous emission rates, and that if they were lower, then the
dropoff would follow the Rabi frequency instead of the spontaneous
emission rates.

\end{document}
